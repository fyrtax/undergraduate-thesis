\documentclass{sprawozdanie-agh}

\usepackage[utf8]{inputenc}
\usepackage{listings}

%\usepackage{utopia} % sans serif
%\usepackage{lmodern} % serif
%\renewcommand{\familydefault}{\sfdefault} % serif

\makeatletter

\begin{document}

\przedmiot{Ważny przedmiot\\Grupa 2}
\tytul{Sprawozdanko}
\podtytul{Niesamowity temat zadania}
\kierunek{Informatyka Geoprzestrzenna}
\rok{Rok 1}
\autor{Janko Muzykant}
%\miasto{Kraków}
%\data{\today}

\stronatytulowa{}
\spistresci{}

\section{Wstęp}
Opis
\subsection{Podpunkt}
Wypis
\subsubsection{Pod podpunkt}
Odpis

\section{Jakiś punkt}
Napis

% bibliografia test
% używa się pliku .bib
% następnie \cite{} - automatycznie dodaje cytowany dokument do bibliografi
\LaTeX{} \cite{latex2e} is a set of macros built atop \TeX{} \cite{texbook}.
test \cite{einstein} and \cite{knuthwebsite}

\bibliographystyle{plain} % We choose the "plain" reference style

\bibliography{refs} % Entries are in the refs.bib file
% end

\end{document}