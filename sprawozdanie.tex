\documentclass{sprawozdanie-agh}
\usepackage{hyperref}

\usepackage[utf8]{inputenc}
\usepackage{listings}

%\usepackage{utopia} % sans serif
%\usepackage{lmodern} % serif
\renewcommand{\familydefault}{\sfdefault} % serif

\makeatletter

\begin{document}

\tytul{Implementacja przeglądarki modeli BIM w oparciu o standardy otwarte}
\podtytul{Implementation of a BIM model viewer based on open standards}
\kierunek{Informatyka Geoprzestrzenna} 
\autor{\textbf{Opiekun pracy: }dr hab. inż. Tomasz Owerko \newline\textbf{Autorzy: }Chłopaki}
\przedmiot{}
%\miasto{Kraków}
%\data{\today}

\stronatytulowa{}
\newpage
To poniżej to jakieś brednie, ja się pod tym nie podpisuję (Szymon):
\url{https://home.agh.edu.pl/~kwant/wordpress/wp-content/uploads/2022/09/Praca_Inzynierska.pdf}
\url{https://home.agh.edu.pl/~romus/Laboratorium%20Aparatury%20Automatyzacji/Dokumentacja%20techniczna%20i%20oprogramowanie/Cwiczenie%201/SadyDamian%20-Stanowisko%20do%20kontroli%20ci%9Cnienia.pdf}
Uprzedzony o odpowiedzialności karnej na podstawie art. 115 ust. 1 i 2 ustawy z dnia 4
lutego 1994 r. o prawie autorskim i prawach pokrewnych (t.j. Dz.U. z 2006 r. Nr 90,
poz. 631 z późn. zm.): „ Kto przywłaszcza sobie autorstwo albo wprowadza w błąd co
do autorstwa całości lub części cudzego utworu albo artystycznego wykonania, podlega
grzywnie, karze ograniczenia wolności albo pozbawienia wolności do lat 3. Tej samej
karze podlega, kto rozpowszechnia bez podania nazwiska lub pseudonimu twórcy cudzy
utwór w wersji oryginalnej albo w postaci opracowania, artystyczne wykonanie albo
publicznie zniekształca taki utwór, artystyczne wykonanie, fonogram, wideogram lub
nadanie.”, a także uprzedzony o odpowiedzialności dyscyplinarnej na podstawie art. 211
ust. 1 ustawy z dnia 27 lipca 2005 r. Prawo o szkolnictwie wyższym (t.j. Dz. U. z 2012
r. poz. 572, z późn. zm.) „Za naruszenie przepisów obowiązujących w uczelni oraz za
czyny uchybiające godności studenta student ponosi odpowiedzialność dyscyplinarną
przed komisją dyscyplinarną albo przed sądem koleżeńskim samorządu studenckiego,
zwanym dalej „sądem koleżeńskim”, oświadczam, że niniejszą pracę dyplomową
wykonałem(-am) osobiście i samodzielnie i że nie korzystałem(-am) ze źródeł innych
niż wymienione w pracy


\newpage
\spistresci{}

\section{Wstęp}
Opinie o całej pracy: 2-3 strony

Google scholar
Science direct
Podawać zawsze kiedy czytalismy ten artykuł
od jednej do 2 stron

\begin{itemize}
    \item BIM digital twin
    \item 
    \url{https://www.mdpi.com/2071-1050/15/13/10462} -mam cytaty PK
    \item building smart o ifc
    \item \url{https://repositum.tuwien.at/bitstream/20.500.12708/192611/5/Eichler-2024-BIMcert%20Handbook%20Basic%20Knowledge%20openBIM-vor.pdf}
    \item \url{https://standards.buildingsmart.org/IFC/RELEASE/IFC4_3/HTML/lexical/IfcWall.htm}
    \item science direct
    \item \url{https://www.sciencedirect.com/science/article/pii/S0926580525001979} - przeczytane mam cytaty PK
    \item \url{https://www.sciencedirect.com/journal/automation-in-construction}
    \item \url {https://www.sciencedirect.com/journal/automation-in-construction}
    \item \url{https://www.sciencedirect.com/science/article/pii/S2467967425000716}
    \item \url{http://sciencedirect.com/science/article/pii/S0926580525003437}
    \item \url{https://www.sciencedirect.com/science/article/pii/S0264837724003211}
    \item xeokit
    \item \url{https://xeokit.github.io/xeokit-sdk/docs/}
    \item \url{https://xeokit.io/}
    \item That open company
    \item \url{https://docs.thatopen.com/intro}
    \item \url{https://github.com/ThatOpen/}
    \item Google scholar
    \item \url{https://ieeexplore.ieee.org/abstract/document/6354353}
    \item \url{https://xeokit.notion.site/Converting-IFC-Models-to-XKT-using-3rd-Party-Open-Source-Tools-c373e48bc4094ff5b6e5c5700ff580ee}
\end{itemize} 

\section{Wprowadzenie - badanie literaturowe}

W tym rozdziale opisane zostaną głównie techniki wykorzystywane w projekcie.

\subsection{Definicja i rozwój technologii BIM}

\subsubsection{Historia i koncepcja BIM}

\subsubsection{Bim}
\subsubsection{Standardy openBIM i building SMART}

\subsection{Cyfrowe Bliźniaki(Digital Twins)}
\subsubsection{Digital Twins}
\subsubsection{Wykorzystanie bliźniaków w budownictwie}
\subsubsection{Przykłady I guess}

\subsection{Standard IFC}
\subsubsection{Historia IFC}
\subsubsection{Struktura i semantyka modelu IFC}
\subsubsection{zalety i wady IFC}

\subsection{Przeglądarki i narzędzia do modeli BIM}
\subsubsection{That open}
\subsubsection{Xeokit}
\subsubsection{Inne open source}
\subsubsection{Porównanie}

\subsection{Podsumowanie literatury}
\subsubsection{Wnioski}
\subsubsection{}
\subsubsection{}

\section{Podstawy technologiczne}
pisze tu bo nwm gdzie: napiać że nauczyłem się nie aktualizować bibliotek podczas tworzenia projektu
\begin{itemize}
    \item podstawy informatyczne
    \
    \item technologia bim 
    \item co to framework
    \item co to react po co wybralismy
    \item jakis schemat semantyczny itd
    \item jak dziala kod
\end{itemize}
\subsection{Architektura aplikacji webowej}
Tutaj drzewko naszej aplikacji
\subsubsection{Frontend}
\subsubsection{Backend}
\subsubsection{Baza datnych}
\subsubsection{API}

\subsection{Wybór technolgii}
\subsubsection{React}
\subsubsection{Vite js}
\subsubsection{Python}
\subsubsection{Slinik xeo}
\subsubsection{}

\subsection{Porównanie ThatOpen vs Xeokit}
\subsubsection{IFC a XKT}
\subsubsection{Kompilacja i prędkość otwierania projektów }

\subsection{System kontroli wersji - Git i Github}
\subsubsection{Stworzenie repo i tworzenie subrepo}
\subsubsection{Projekt z Backlog}
\subsubsection{Praca z PR i code review(Git Flow)}
\subsubsection{Zarzadzanie wersjami i gałęziami}

\section{Metody}

Z czego korzystalismy, co finalnie zadzialalo 
co nie zadzilaalo dlaczego czegos nie rozwijalismy itd co robilsmy wokol jakich jezykow uzywalismy ale nie pomogly

\subsection{Dwie przeglądarki}
\subsubsection{Przeglądarka ThatOpen}
\begin{itemize}
    \item Integracja loadera IFC
    \item Wydajność
    \item Ograniczenia biblioteki
\end{itemize}
\subsubsection{Przeglądarka Xeokit}
\begin{itemize}
    \item Xeo conventer 
    \item Wydajność
    \item Ograniczenia 
\end{itemize}

\subsection{Proces implementacji przegląarki BIM}
\subsubsection{Tworzenie Frontend}
\subsubsection{Tworzenie xeokit}
\subsubsection{Tworzenie API}
\subsubsection{Tworzenie bazy danych SQL}

\subsection{Dlaczego Xeokit}


\subsection{Testy}
\subsubsection{Testy ładowania modeli}
\subsubsection{Testy API}
\subsubsection{Testy interaktywne}
\subsubsection{}section{Testy wydajnościowe}

\subsection{Projekt i backlog}
\subsubsection{Tworzenie backlogu}
\subsubsection{Dokumentacja }

\section{Rezultaty}
\begin{itemize}
    \item Screeny
    \item testy!!!
    \item troche kodu
\end{itemize}

\subsection{Działająca przeglądarka BIM}
\subsubsection{Opisy funkcji modułów itd}
\subsubsection{Interfejs}

\subsection{Zrzuty ekranu}
\subsubsection{Strona główna}
\subsubsection{Edycja modeli}
\subsubsection{Wyłączanie i właczanie elementów projektu}

\subsection{Fragmenty kodu}
\subsubsection{API}

\section{Dyskusja}
\begin{itemize}
    \item koniec - wnioski np. wypunktowanie co robiliśmy korzystalismy z that open ale nie dziala wiec xeokit 
    \item podsumowanie napisane prozą co zrobiliśmy co jest najbardziej udane i np. sugestie co dalej zrobic 
\end{itemize}

\subsection{Analiza zrealizowanych celów}
\subsection{Wnioski praktyczne}
\subsection{Propozycja dalszego rozwoju}

\section{koniec koniec (literatura)}
literature cytujemy na końcu cytatu numer i numer na literaturze przy żródłach internetowych date kiedy korzystalismy 

\section{Co dodac to przegladarki}
\begin{itemize}
    \item jak najwiecej komponentow
    \item moze by duzo takich aniumacji na stronie 
    \item duzo smaczkow 
    \item linki do stron roznych naszych profili itd zeby sie chwalic znajomoscia z wszystkim costam o katedrze z ktora robilismy
\end{itemize}

% bibliografia test
% używa się pliku .bib
% następnie \cite{} - automatycznie dodaje cytowany dokument do bibliografi
\LaTeX{} \cite{latex2e} is a set of macros built atop \TeX{} \cite{texbook}.
test \cite{einstein} and \cite{knuthwebsite}

\bibliographystyle{plain} % We choose the "plain" reference style

\bibliography{refs} % Entries are in the refs.bib file

\end{document}