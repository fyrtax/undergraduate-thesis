\documentclass{sprawozdanie-agh}
\usepackage{hyperref}

\usepackage[utf8]{inputenc}
\usepackage{listings}

%\usepackage{utopia} % sans serif
%\usepackage{lmodern} % serif
\renewcommand{\familydefault}{\sfdefault} % serif

\makeatletter

\begin{document}

\tytul{Implementacja przeglądarki modeli BIM w oparciu o standardy otwarte}
\podtytul{Implementation of a BIM model viewer based on open standards}
\kierunek{Informatyka Geoprzestrzenna}
\autor{Chłopaki}
\przedmiot{}
%\miasto{Kraków}
%\data{\today}

\stronatytulowa{}
\spistresci{}

\section{Wstęp}
Opinie o całej pracy: 2-3 strony

\section{Wprowadzenie - badanie literaturowe}
Google scholar
Science direct
Podawać zawsze kiedy czytalismy ten artykuł
od jednej do 2 stron

\begin{itemize}
    \item building smart o ifc
    \item \url{https://repositum.tuwien.at/bitstream/20.500.12708/192611/5/Eichler-2024-BIMcert%20Handbook%20Basic%20Knowledge%20openBIM-vor.pdf}
    \item \url{https://standards.buildingsmart.org/IFC/RELEASE/IFC4_3/HTML/lexical/IfcWall.htm}
    \item science direct
    \item \url{https://www.sciencedirect.com/science/article/pii/S0926580525001979}
    \item \url{https://www.sciencedirect.com/journal/automation-in-construction}
    \item \url {https://www.sciencedirect.com/journal/automation-in-construction}
    \item \url{https://www.sciencedirect.com/science/article/pii/S2467967425000716}
    \item \url{http://sciencedirect.com/science/article/pii/S0926580525003437}
    \item \url{https://www.sciencedirect.com/science/article/pii/S0264837724003211}
    \item xeokit
    \item \url{https://xeokit.github.io/xeokit-sdk/docs/}
    \item \url{https://xeokit.io/}
    \item That open company
    \item \url{https://docs.thatopen.com/intro}
    \item \url{https://github.com/ThatOpen/}
    \item Google scholar
    \item \url{https://ieeexplore.ieee.org/abstract/document/6354353}
\end{itemize}

\section{Podstawy technologiczne}
\begin{itemize}
    \item podstawy informatyczne
    \item technologia bim 
    \item co to framework
    \item co to react po co wybralismy
    \item jakis schemat semantyczny itd
    \item jak dziala kod
\end{itemize}

\section{Metody}
Z czego korzystalismy, co finalnie zadzialalo 
co nie zadzilaalo dlaczego czegos nie rozwijalismy itd co robilsmy wokol jakich jezykow uzywalismy ale nie pomogly 

\section{Rezultaty}
\begin{itemize}
    \item Screeny
    \item testy!!!
    \item troche kodu
\end{itemize}

\section{Dyskusja}
\begin{itemize}
    \item koniec - wnioski np. wypunktowanie co robiliśmy korzystalismy z that open ale nie dziala wiec xeokit 
    \item podsumowanie napisane prozą co zrobiliśmy co jest najbardziej udane i np. sugestie co dalej zrobic 
\end{itemize}

\section{koniec koniec (literatura)}
literature cytujemy na końcu cytatu numer i numer na literaturze przy żródłach internetowych date kiedy korzystalismy 

\section{Co dodac to przegladarki}
\begin{itemize}
    \item jak najwiecej komponentow
    \item moze by duzo takich aniumacji na stronie 
    \item duzo smaczkow 
    \item linki do stron roznych naszych profili itd zeby sie chwalic znajomoscia z wszystkim costam o katedrze z ktora robilismy
\end{itemize}

% bibliografia test
% używa się pliku .bib
% następnie \cite{} - automatycznie dodaje cytowany dokument do bibliografi
\LaTeX{} \cite{latex2e} is a set of macros built atop \TeX{} \cite{texbook}.
test \cite{einstein} and \cite{knuthwebsite}

\bibliographystyle{plain} % We choose the "plain" reference style

\bibliography{refs} % Entries are in the refs.bib file

\end{document}